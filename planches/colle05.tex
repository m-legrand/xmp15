\documentclass[12pt,a4paper]{report}

\usepackage[utf8]{inputenc}
\usepackage[T1]{fontenc}
\usepackage[francais,frenchb]{babel}
\usepackage[top=1.5cm, bottom=1.5cm, left=2cm, right=2cm]{geometry}
\usepackage{lmodern}
\usepackage{amsmath}
\usepackage{amssymb}
\usepackage{graphicx}
\usepackage{mathrsfs}
\usepackage{amsthm}
\usepackage{stmaryrd}
\usepackage{verbatim}
\usepackage{moreverb}
\usepackage{textcomp}
\usepackage{hyperref} % liens avec \url
\usepackage[normalem]{ulem} % pour barrer avec \sout, et autres effets intéressants

\DeclareMathOperator{\pgcd}{pgcd}
\DeclareMathOperator{\tr}{tr}

\begin{document}

\newcommand{\rA}{\mathcal{A}}
\newcommand{\rB}{\mathcal{B}}
\newcommand{\rC}{\mathcal{C}}
\newcommand{\rG}{\mathcal{G}}
\newcommand{\rF}{\mathcal{F}}
\newcommand{\rO}{\mathcal{O}}
\newcommand{\rP}{\mathcal{P}}
\newcommand{\rH}{\mathcal{H}}
\newcommand{\rR}{\mathcal{R}}
\newcommand{\rE}{\mathcal{E}}
\newcommand{\rL}{\mathcal{L}}
\newcommand{\rM}{\mathcal{M}}
\newcommand{\bK}{\mathbb{K}}
\newcommand{\bQ}{\mathbb{Q}}
\newcommand{\bR}{\mathbb{R}}
\newcommand{\bZ}{\mathbb{Z}}
\newcommand{\bN}{\mathbb{N}}
\newcommand{\bC}{\mathbb{C}}
\newcommand{\bP}{\mathbb{P}}
\newcommand{\bF}{\mathbb{F}}

\ifpdf
\DeclareGraphicsExtensions{.pdf, .jpg, .tif}
\else
\DeclareGraphicsExtensions{.eps, .jpg}
\fi

%%%%%%%%%%%%%%%%%%%%%%%%%%%%%%%%%%%%%%%%%%%%%%%%%%%%%%%%
\pagestyle{empty} %Pas de numérotation

\noindent \textsc{Institution Sainte-Marie}\\
Colle de Mathématiques n°5 (XMP)\\
Lundi 12 octobre 2015

% Sources : 
% Tout-en-Un Dunod - Chapitre 11

\subsection*{\underline{Sujet 1}}
\paragraph*{Question (Cours)} % Cours
Théorème de Heine.

\paragraph*{Exercice (Lemme de Croft)} % Exo 6
Soit $f : \bR_+ \rightarrow \bR$ telle que $\forall x >0, \lim\limits_{n\rightarrow\infty}f(nx)=0$.
\begin{itemize}
\item Montrer que si $f$ est uniformément continue, alors $\lim\limits_{x\rightarrow\infty}f(x)=0$.
\item Discuter du cas où $f$ est seulement continue.
\end{itemize}

\subsection*{\underline{Sujet 2}}
\paragraph*{Question (Cours)} % Cours
Définition d'une application continue. Réciproque d'un ouvert (\emph{resp.} d'un fermé) par une application continue.

\paragraph*{Exercice} % Exo 13
\begin{itemize}
\item Montrer que l'ensemble $\rO(n)$ des matrices orthogonales sur $\bR^n$ est compact.
\item Montrer qu'il existe une bijection continue à réciproque continue entre $\rG\rL_n(\bR)$ et $\bR^{\frac{n(n+1)}{2}}\times \rO(n)$. On dit alors que ces deux espaces sont \emph{homéomorphes}.
\end{itemize}


\subsection*{\underline{Sujet 3}}
\paragraph*{Question (Cours)} % Cours
Démontrer la continuité du déterminant.

\paragraph*{Exercice} % Exo 18
Soient $E = \{u \in \bR^\bN \ |\ \sum\limits_{k=0}^\infty u_k^2<\infty\}$ et $e_n=(\delta_{k=n})_{k\in\bN}$.
\begin{itemize}
\item Montrer que $A=\{0\}\cup\{\frac{1}{n}e_n \ |\ n\in\bN\}$ est compact.
\item Montrer que l'enveloppe convexe de $A$ dans $E$ (c'est à dire la plus petite partie convexe de $E$ contenant $A$) n'est pas compacte.
\end{itemize}

\begin{center}
\url{https://github.com/m-legrand/xmp15}
\end{center}

\end{document}