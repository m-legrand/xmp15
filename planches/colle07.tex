\documentclass[12pt,a4paper]{report}

\usepackage[utf8]{inputenc}
\usepackage[T1]{fontenc}
\usepackage[francais,frenchb]{babel}
\usepackage[top=1.5cm, bottom=1.5cm, left=2cm, right=2cm]{geometry}
\usepackage{lmodern}
\usepackage{amsmath}
\usepackage{amssymb}
\usepackage{graphicx}
\usepackage{mathrsfs}
\usepackage{amsthm}
\usepackage{stmaryrd}
\usepackage{verbatim}
\usepackage{moreverb}
\usepackage{textcomp}
\usepackage{hyperref} % liens avec \url
\usepackage[normalem]{ulem} % pour barrer avec \sout, et autres effets intéressants

\DeclareMathOperator{\pgcd}{pgcd}
\DeclareMathOperator{\tr}{tr}

\begin{document}

\newcommand{\rA}{\mathcal{A}}
\newcommand{\rB}{\mathcal{B}}
\newcommand{\rC}{\mathcal{C}}
\newcommand{\rG}{\mathcal{G}}
\newcommand{\rF}{\mathcal{F}}
\newcommand{\rO}{\mathcal{O}}
\newcommand{\rP}{\mathcal{P}}
\newcommand{\rH}{\mathcal{H}}
\newcommand{\rR}{\mathcal{R}}
\newcommand{\rE}{\mathcal{E}}
\newcommand{\rL}{\mathcal{L}}
\newcommand{\rM}{\mathcal{M}}
\newcommand{\bK}{\mathbb{K}}
\newcommand{\bQ}{\mathbb{Q}}
\newcommand{\bR}{\mathbb{R}}
\newcommand{\bZ}{\mathbb{Z}}
\newcommand{\bN}{\mathbb{N}}
\newcommand{\bC}{\mathbb{C}}
\newcommand{\bP}{\mathbb{P}}
\newcommand{\bF}{\mathbb{F}}

\ifpdf
\DeclareGraphicsExtensions{.pdf, .jpg, .tif}
\else
\DeclareGraphicsExtensions{.eps, .jpg}
\fi

%%%%%%%%%%%%%%%%%%%%%%%%%%%%%%%%%%%%%%%%%%%%%%%%%%%%%%%%
\pagestyle{empty} %Pas de numérotation

\noindent \textsc{Institution Sainte-Marie}\\
Colle de Mathématiques n°7 (XMP)\\
Lundi 9 novembre 2015

% Sources : 
% Oraux X-ENS - Analyse 2

\subsection*{\underline{QC 1}}
Limite uniforme de fonctions continues.

\subsection*{\underline{QC 2}}
Intégration d'une limite uniforme sur un segment.

\subsection*{\underline{QC 3}}
Dérivation d'une suite de fonctions.

\subsection*{\underline{Exercice 1}}
Soit $f : \bR \rightarrow \bR$ une fonction $1$-périodique définie par $\forall x \in [-\frac{1}{2}, \frac{1}{2}], f(x) = x^2$. Montrer que $$\forall x \in \bR, \sum\limits_{n=-\infty}^\infty 2^nf\left(\frac{x}{2^n}\right) = |x|.$$

\subsection*{\underline{Exercice 2}}
Soit $f \in \rC^0(\bR, \bR)$ telle que $\exists \ a>0 : \forall x,y, |f(x+y)-f(x)-f(y)| \leq a$. Montrer que $\forall x, g_n(x):=\frac{f(2^nx)}{2^n}$ admet une limite $g(x)$, que $g$ est linéaire, continue, et que $f-g$ est bornée.

\vfill

\begin{center}
\url{https://github.com/m-legrand/xmp15}
\end{center}

\end{document}