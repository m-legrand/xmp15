\documentclass[12pt,a4paper]{report}

\usepackage[utf8]{inputenc}
\usepackage[T1]{fontenc}
\usepackage[francais,frenchb]{babel}
\usepackage[top=1.5cm, bottom=1.5cm, left=2cm, right=2cm]{geometry}
\usepackage{lmodern}
\usepackage{amsmath}
\usepackage{amssymb}
\usepackage{graphicx}
\usepackage{mathrsfs}
\usepackage{amsthm}
\usepackage{stmaryrd}
\usepackage{verbatim}
\usepackage{moreverb}
\usepackage{textcomp}
\usepackage{hyperref} % liens avec \url
\usepackage[normalem]{ulem} % pour barrer avec \sout, et autres effets intéressants

\DeclareMathOperator{\pgcd}{pgcd}
\DeclareMathOperator{\tr}{tr}

\begin{document}

\newcommand{\rA}{\mathcal{A}}
\newcommand{\rB}{\mathcal{B}}
\newcommand{\rC}{\mathcal{C}}
\newcommand{\rG}{\mathcal{G}}
\newcommand{\rF}{\mathcal{F}}
\newcommand{\rO}{\mathcal{O}}
\newcommand{\rP}{\mathcal{P}}
\newcommand{\rH}{\mathcal{H}}
\newcommand{\rR}{\mathcal{R}}
\newcommand{\rE}{\mathcal{E}}
\newcommand{\rL}{\mathcal{L}}
\newcommand{\rM}{\mathcal{M}}
\newcommand{\bK}{\mathbb{K}}
\newcommand{\bQ}{\mathbb{Q}}
\newcommand{\bR}{\mathbb{R}}
\newcommand{\bZ}{\mathbb{Z}}
\newcommand{\bN}{\mathbb{N}}
\newcommand{\bC}{\mathbb{C}}
\newcommand{\bP}{\mathbb{P}}
\newcommand{\bF}{\mathbb{F}}

\ifpdf
\DeclareGraphicsExtensions{.pdf, .jpg, .tif}
\else
\DeclareGraphicsExtensions{.eps, .jpg}
\fi

%%%%%%%%%%%%%%%%%%%%%%%%%%%%%%%%%%%%%%%%%%%%%%%%%%%%%%%%
\pagestyle{empty} %Pas de numérotation

\noindent \textsc{Institution Sainte-Marie}\\
Colle de Mathématiques n°10 (XMP)\\
Lundi 30 novembre 2015

% Source : 
% http://www.maths-france.fr/MathSpe/Exercices/

\subsection*{\underline{Question 1}}

Soit $A\in M_n(\bC)$. Montrer qu'il existe $p_0\in\bN$ tel que $\sum\limits_{k=0}^{p_0}\frac{A^k}{k!}$ soit inversible.

\subsection*{\underline{Question 2}}

Soit $A=\begin{pmatrix}
3 & 2 & 2\\
1 & 0 & 1\\
-1 & 1 & 0\\
\end{pmatrix}$. Calculer $\exp(tA), t\in\bR$.

\subsection*{\underline{Question 3}}

Soit $A=\begin{pmatrix}
4 & 1 & 1\\
6 & 4 & 2\\
-10 & -4 & -2\\
\end{pmatrix}$. Calculer $\exp(tA), t\in\bR$.

\subsection*{\underline{Question 4}}

Soit $A\in M_n(\bC)$. Calculer $\lim\limits_{p\rightarrow +\infty}\left(I_n+\frac{A}{p}\right)^p$.

\subsection*{\underline{Question 5}}

Soit $A\in M_n(\bC)$. Montrer que $\exp(A)$ est un polynôme en $A$.

\subsection*{\underline{Problème}}

\begin{itemize}
\item[1.]
\begin{itemize}
\item[a)] Soit $\omega\in\bR^3$. Montrer que $f_\omega : x \mapsto \omega \wedge x$ est un morphisme anti-symétrique sur $\bR^3$.
\item[b)] Montrer que tout morphisme anti-symétrique s'écrit sous la forme d'un unique $f_\omega$.
\end{itemize}
\item[2.] Soit $\omega\in\bR^3$. Montrer que $\exp(f_\omega)$ est une rotation, et en donner l'axe et l'angle.
\end{itemize}

\vfill

\begin{center}
\url{https://github.com/m-legrand/xmp15}
\end{center}

\end{document}