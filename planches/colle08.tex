\documentclass[12pt,a4paper]{report}

\usepackage[utf8]{inputenc}
\usepackage[T1]{fontenc}
\usepackage[francais,frenchb]{babel}
\usepackage[top=1.5cm, bottom=1.5cm, left=2cm, right=2cm]{geometry}
\usepackage{lmodern}
\usepackage{amsmath}
\usepackage{amssymb}
\usepackage{graphicx}
\usepackage{mathrsfs}
\usepackage{amsthm}
\usepackage{stmaryrd}
\usepackage{verbatim}
\usepackage{moreverb}
\usepackage{textcomp}
\usepackage{hyperref} % liens avec \url
\usepackage[normalem]{ulem} % pour barrer avec \sout, et autres effets intéressants

\DeclareMathOperator{\pgcd}{pgcd}
\DeclareMathOperator{\tr}{tr}

\begin{document}

\newcommand{\rA}{\mathcal{A}}
\newcommand{\rB}{\mathcal{B}}
\newcommand{\rC}{\mathcal{C}}
\newcommand{\rG}{\mathcal{G}}
\newcommand{\rF}{\mathcal{F}}
\newcommand{\rO}{\mathcal{O}}
\newcommand{\rP}{\mathcal{P}}
\newcommand{\rH}{\mathcal{H}}
\newcommand{\rR}{\mathcal{R}}
\newcommand{\rE}{\mathcal{E}}
\newcommand{\rL}{\mathcal{L}}
\newcommand{\rM}{\mathcal{M}}
\newcommand{\bK}{\mathbb{K}}
\newcommand{\bQ}{\mathbb{Q}}
\newcommand{\bR}{\mathbb{R}}
\newcommand{\bZ}{\mathbb{Z}}
\newcommand{\bN}{\mathbb{N}}
\newcommand{\bC}{\mathbb{C}}
\newcommand{\bP}{\mathbb{P}}
\newcommand{\bF}{\mathbb{F}}

\ifpdf
\DeclareGraphicsExtensions{.pdf, .jpg, .tif}
\else
\DeclareGraphicsExtensions{.eps, .jpg}
\fi

%%%%%%%%%%%%%%%%%%%%%%%%%%%%%%%%%%%%%%%%%%%%%%%%%%%%%%%%
\pagestyle{empty} %Pas de numérotation

\noindent \textsc{Institution Sainte-Marie}\\
Colle de Mathématiques n°8 (XMP)\\
Lundi 16 novembre 2015

% Source : 
% Les maths en tête - Analyse

\subsection*{\underline{Sujet 1}}

\paragraph*{Question de cours}
Lemme d'Abel.

\paragraph*{Question (Formule de Cauchy)} Soit $\sum a_nz^n$ une série entière de rayon de convergence $R>0$ et $f$ la somme de cette série sur son disque de convergence. Montrer que 
$$\forall r\in ]0,R[, n\in\bN, \int_0^{2\pi}f(re^{i\theta})e^{-ni\theta}d\theta.$$

\paragraph*{Exercice}
Donner le rayon de convergence $R$, puis la somme des séries entières suivantes :
\begin{itemize}
\item[a)] $\sum\limits_{n\in\bN}n^2x^n$;
\item[b)] $\sum\limits_{n\in\bN}\frac{x^n}{2n+1}$ (sur $\bR_+^*$);
\item[c)] $\sum\limits_{n\in\bN^*}\frac{x^n}{n(n+2)}$;
\item[d)] $\sum\limits_{n\in\bN}\frac{x^n}{n!}cos(n\theta)$.
\end{itemize}

\subsection*{\underline{Sujet 2}}

\paragraph*{Question de cours}
Produit de Cauchy.

\paragraph*{Exercice}
Après en avoir justifié l'existence, développer en série entière la fonction 
$$\begin{array}{ccccc}
f & : & ]-1,1[ & \to & \bR \\
 & & x & \mapsto & (\arcsin x)^2 .\\
\end{array}$$

\subsection*{\underline{Sujet 3}}

\paragraph*{Question de cours}
Dérivée(s) d'une série entière.

\paragraph*{Exercice}
$$\begin{array}{ccccc}
f & : & ]0,1[ & \to & \bR \\
 & & x & \mapsto & \frac{\arcsin \sqrt{x}}{\sqrt{x(1-x)}} \\
\end{array}$$
Montrer que $f$ coïncide sur $]0,1[$ avec la somme d'une série entière, et en calculer les coefficients.

\vfill

\begin{center}
\url{https://github.com/m-legrand/xmp15}
\end{center}

\end{document}