\documentclass[12pt,a4paper]{report}

\usepackage[utf8]{inputenc}
\usepackage[T1]{fontenc}
\usepackage[francais,frenchb]{babel}
\usepackage[top=1.5cm, bottom=1.5cm, left=2cm, right=2cm]{geometry}
\usepackage{lmodern}
\usepackage{amsmath}
\usepackage{amssymb}
\usepackage{graphicx}
\usepackage{mathrsfs}
\usepackage{amsthm}
\usepackage{stmaryrd}
\usepackage{verbatim}
\usepackage{moreverb}
\usepackage{textcomp}
\usepackage{hyperref} % liens avec \url
\usepackage[normalem]{ulem} % pour barrer avec \sout, et autres effets intéressants

\DeclareMathOperator{\pgcd}{pgcd}
\DeclareMathOperator{\tr}{tr}

\begin{document}

\newcommand{\rA}{\mathcal{A}}
\newcommand{\rB}{\mathcal{B}}
\newcommand{\rC}{\mathcal{C}}
\newcommand{\rG}{\mathcal{G}}
\newcommand{\rF}{\mathcal{F}}
\newcommand{\rP}{\mathcal{P}}
\newcommand{\rH}{\mathcal{H}}
\newcommand{\rR}{\mathcal{R}}
\newcommand{\rE}{\mathcal{E}}
\newcommand{\rL}{\mathcal{L}}
\newcommand{\rM}{\mathcal{M}}
\newcommand{\bK}{\mathbb{K}}
\newcommand{\bQ}{\mathbb{Q}}
\newcommand{\bR}{\mathbb{R}}
\newcommand{\bZ}{\mathbb{Z}}
\newcommand{\bN}{\mathbb{N}}
\newcommand{\bC}{\mathbb{C}}
\newcommand{\bP}{\mathbb{P}}
\newcommand{\bF}{\mathbb{F}}

\ifpdf
\DeclareGraphicsExtensions{.pdf, .jpg, .tif}
\else
\DeclareGraphicsExtensions{.eps, .jpg}
\fi

%%%%%%%%%%%%%%%%%%%%%%%%%%%%%%%%%%%%%%%%%%%%%%%%%%%%%%%%
\pagestyle{empty} %Pas de numérotation

\noindent \textsc{Institution Sainte-Marie}\\
Colle de Mathématiques n°4 (XMP)\\
Lundi 5 octobre 2015

% Sources : 
% [1] Les maths en tête (Analyse) - Xavier Gourdon - Ellipses

\subsection*{\underline{Sujet 1}}
\paragraph*{Question (Cours)} % Cours
Montrer qu'une suite possédant au moins deux valeurs d'adhérence diverge. Une suite possédant une unique valeur d'adhérence converge-t-elle ? 

\paragraph*{Exercice} % [1] Chapitre 1.5 - Exercice 7
Soit $\varphi$ une forme linéaire sur un $\bR$-evn $E$. Montrer que $\varphi$ est continue si et seulement si son noyau est une partie fermée de $E$.

\subsection*{\underline{Sujet 2}} % [1] Chapitre 1.5 - Exercice 6

\paragraph*{(Théorème de Carathéodory)} 
Soit $E$ un $\bR$-ev de dimension $n\in \bN^*$ et $x$ le barycentre de $p$ vecteurs de $E$. Montrer que $x$ est le barycentre d'au plus $n+1$ de ces vecteurs.

\subsection*{\underline{Sujet 3}}

\paragraph*{Question} % [1] Chapitre 1.5 - Exercice 6
Soient $E$ un e.v.n. et $C$ un convexe de $E$. Montrer que l'intérieur et l'adhérence de $C$ sont convexes.

\paragraph*{Exercice} % [1] Chapitre 1.5 - Exercice 9
Soit $E$ un $\bR$-evn de dimension infinie.  Montrer que la boule $B_f(0,1)$ ne peut pas être incluse dans une réunion finie de boules ouvertes de rayon $1$.

\subsection*{\underline{Problème bonus}} % [1] Chapitre 1.5 - Exercice 6

\paragraph*{Application du théorème de Carathéodory}
On suppose vrai le résultat du \textbf{\underline{Sujet 2}}. 
On donne (ou rappelle) les définitions suivantes :
\begin{itemize}
\item L'\emph{enveloppe convexe} $Conv(A)$ dans $E$ d'un sous-ensemble $A$ de $E$ est l'intersection de l'ensemble des parties convexes de $E$ contenant $A$.
\item On dit qu'un ensemble $C$ est \emph{compact} si toute suite d'éléments de $C$ admet une valeur d'adhérence dans $C$.
\end{itemize}
Montrer que l'enveloppe convexe $Conv(A)$ d'une partie compacte $A$ d'un $\bR$-evn $E$ de dimension finie est compacte.
\vfill

\begin{center}
\url{https://github.com/m-legrand/xmp15}
\end{center}

\end{document}