\documentclass[12pt,a4paper]{report}

\usepackage[utf8]{inputenc}
\usepackage[T1]{fontenc}
\usepackage[francais,frenchb]{babel}
\usepackage[top=1.5cm, bottom=1.5cm, left=2cm, right=2cm]{geometry}
\usepackage{lmodern}
\usepackage{amsmath}
\usepackage{amssymb}
\usepackage{graphicx}
\usepackage{mathrsfs}
\usepackage{amsthm}
\usepackage{stmaryrd}
\usepackage{verbatim}
\usepackage{moreverb}
\usepackage{textcomp}
\usepackage{hyperref} % liens avec \url
\usepackage{ulem} % pour barrer avec \sout

\DeclareMathOperator{\pgcd}{pgcd}

\begin{document}

\newcommand{\rA}{\mathcal{A}}
\newcommand{\rB}{\mathcal{B}}
\newcommand{\rC}{\mathcal{C}}
\newcommand{\rG}{\mathcal{G}}
\newcommand{\rF}{\mathcal{F}}
\newcommand{\rP}{\mathcal{P}}
\newcommand{\rH}{\mathcal{H}}
\newcommand{\rR}{\mathcal{R}}
\newcommand{\rE}{\mathcal{E}}
\newcommand{\rL}{\mathcal{L}}
\newcommand{\rM}{\mathcal{M}}
\newcommand{\bK}{\mathbb{K}}
\newcommand{\bQ}{\mathbb{Q}}
\newcommand{\bR}{\mathbb{R}}
\newcommand{\bZ}{\mathbb{Z}}
\newcommand{\bN}{\mathbb{N}}
\newcommand{\bC}{\mathbb{C}}
\newcommand{\bP}{\mathbb{P}}
\newcommand{\bF}{\mathbb{F}}

\ifpdf
\DeclareGraphicsExtensions{.pdf, .jpg, .tif}
\else
\DeclareGraphicsExtensions{.eps, .jpg}
\fi

%%%%%%%%%%%%%%%%%%%%%%%%%%%%%%%%%%%%%%%%%%%%%%%%%%%%%%%%
\pagestyle{empty} %Pas de numérotation

\noindent \textsc{Institution Sainte-Marie}\\
Colle de Mathématiques n°1 (XMP)\\
Lundi 14 septembre 2015\\

Dans ces exercices, les notions de ``simple'' ou ``double'' à propos d'une valeur propre réfèrent à leur ordre de multiplicité dans le polynôme caractéristique.

% Source : Oraux X-ENS Algèbre 2 - CASSINI

\subsection*{\underline{Exercice 1}} % 2.3
On pose $A_1= \begin{pmatrix} 1 & 1 \\ 1 & -1 \end{pmatrix}$ et $A_{n+1}=\begin{pmatrix} A_n & A_n \\ A_n & -A_n \end{pmatrix}$. Trouver les valeurs propres de $A_n$.

\subsection*{\underline{Exercice 2}} % 2.2
Soient $A,B,C \in M_2(\bK)$. Mq $\exists$ une combinaison linéaire à coefficients non tous nuls de ces matrices admettant une valeur propre double.

\subsection*{\underline{Exercice 3}} % 2.1
Soit $\bK$ un corps, $\lambda\in\bK, A\in M_n(\bK)$, et $X,Y\in M_{n,1}(\bK)$ tels que :
\begin{enumerate}
\item $AX=\lambda X$ ;
\item $~^tYA = \lambda ~^tY$ ;
\item $~^tY=X \neq 0$ ;
\item $rg(A-\lambda I_n)=n-1$.
\end{enumerate}

Montrer que $\lambda$ est une valeur propre simple.

\subsection*{\underline{Exercice 4}} % 2.4
Soit $A\in M_n(\bC)$. On définit une suite $(A_k)$ avec $A_0=A$ et $A_{k+1}=A(A_k-\frac{1}{k+1}tr(A_k)I_n)$. Montrer que $A_n=0$.

\vfill

\begin{center}
\url{https://github.com/m-legrand/xmp15}
\end{center}

\end{document}