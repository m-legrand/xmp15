\documentclass[12pt,a4paper]{report}

\usepackage[utf8]{inputenc}
\usepackage[T1]{fontenc}
\usepackage[francais,frenchb]{babel}
\usepackage[top=1.5cm, bottom=1.5cm, left=2cm, right=2cm]{geometry}
\usepackage{lmodern}
\usepackage{amsmath}
\usepackage{amssymb}
\usepackage{graphicx}
\usepackage{mathrsfs}
\usepackage{amsthm}
\usepackage{stmaryrd}
\usepackage{verbatim}
\usepackage{moreverb}
\usepackage{textcomp}
\usepackage{hyperref} % liens avec \url
\usepackage[normalem]{ulem} % pour barrer avec \sout, et autres effets intéressants

\DeclareMathOperator{\pgcd}{pgcd}
\DeclareMathOperator{\tr}{tr}

\begin{document}

\newcommand{\rA}{\mathcal{A}}
\newcommand{\rB}{\mathcal{B}}
\newcommand{\rC}{\mathcal{C}}
\newcommand{\rG}{\mathcal{G}}
\newcommand{\rF}{\mathcal{F}}
\newcommand{\rO}{\mathcal{O}}
\newcommand{\rP}{\mathcal{P}}
\newcommand{\rH}{\mathcal{H}}
\newcommand{\rR}{\mathcal{R}}
\newcommand{\rE}{\mathcal{E}}
\newcommand{\rL}{\mathcal{L}}
\newcommand{\rM}{\mathcal{M}}
\newcommand{\bK}{\mathbb{K}}
\newcommand{\bQ}{\mathbb{Q}}
\newcommand{\bR}{\mathbb{R}}
\newcommand{\bZ}{\mathbb{Z}}
\newcommand{\bN}{\mathbb{N}}
\newcommand{\bC}{\mathbb{C}}
\newcommand{\bP}{\mathbb{P}}
\newcommand{\bF}{\mathbb{F}}

\ifpdf
\DeclareGraphicsExtensions{.pdf, .jpg, .tif}
\else
\DeclareGraphicsExtensions{.eps, .jpg}
\fi

%%%%%%%%%%%%%%%%%%%%%%%%%%%%%%%%%%%%%%%%%%%%%%%%%%%%%%%%
\pagestyle{empty} %Pas de numérotation

\noindent \textsc{Institution Sainte-Marie}\\
Colle de Mathématiques n°9 (XMP)\\
Lundi 23 novembre 2015

% Source : 
% http://mp.cpgedupuydelome.fr

\subsection*{\underline{Sujet 1}}

\paragraph*{Question de cours}
Formule de Taylor-Young.

\paragraph*{Exercice}
Soit $f : \bR \rightarrow E$ de classe $\rC^n$. Donner l'existence et la valeur de 
$\lim\limits_{h\rightarrow 0}\frac{1}{h^n}\sum\limits_{k=0}^n(-1)^k\begin{pmatrix} n\\ k \end{pmatrix}f(kh).$

\subsection*{\underline{Sujet 2}}

\paragraph*{Question de cours}
Dérivabilité et dérivée de $f\circ\varphi$ où $\varphi$ est une fonction réelle de variable réelle et $f$ est une fonction vectorielle.

\paragraph*{Exercice}
Soit $f : [0,1] \rightarrow E$ dérivable à droite en $0$ et vérifiant $f(0)=0$. Donner l'existence et la valeur de 
$\lim\limits_{n\rightarrow +\infty}\sum\limits_{k=0}^nf\left(\frac{k}{n^2}\right).$

\subsection*{\underline{Sujet 3}}

\paragraph*{Question de cours}
Inégalité des accroissements finis pour une fonction vectorielle de classe $\rC^1$.

\paragraph*{Exercice}
On munit $\bR^3$ de sa structure euclidienne orientée usuelle, et on se donne trois fonctions $e_1, e_2, e_3 \in \rC^1(\bR,\bR^3)$ telles que $\forall t \in \bR, (e_1(t), e_2(t), e_3(t))$ est une BOND de $\bR^3$.\\
Exhiber une fonction $\Omega \in \rC^0(\bR, \bR^3)$ telle que $\forall i\in \{1, 2, 3\}, e_i'=\Omega\wedge e_i$.

\vfill

\begin{center}
\url{https://github.com/m-legrand/xmp15}
\end{center}

\end{document}