\documentclass[12pt,a4paper]{report}

\usepackage[utf8]{inputenc}
\usepackage[T1]{fontenc}
\usepackage[francais,frenchb]{babel}
\usepackage[top=1.5cm, bottom=1.5cm, left=2cm, right=2cm]{geometry}
\usepackage{lmodern}
\usepackage{amsmath}
\usepackage{amssymb}
\usepackage{graphicx}
\usepackage{mathrsfs}
\usepackage{amsthm}
\usepackage{stmaryrd}
\usepackage{verbatim}
\usepackage{moreverb}
\usepackage{textcomp}
\usepackage{hyperref} % liens avec \url
\usepackage[normalem]{ulem} % pour barrer avec \sout, et autres effets intéressants

\DeclareMathOperator{\pgcd}{pgcd}
\DeclareMathOperator{\tr}{tr}

\begin{document}

\newcommand{\rA}{\mathcal{A}}
\newcommand{\rB}{\mathcal{B}}
\newcommand{\rC}{\mathcal{C}}
\newcommand{\rG}{\mathcal{G}}
\newcommand{\rF}{\mathcal{F}}
\newcommand{\rP}{\mathcal{P}}
\newcommand{\rH}{\mathcal{H}}
\newcommand{\rR}{\mathcal{R}}
\newcommand{\rE}{\mathcal{E}}
\newcommand{\rL}{\mathcal{L}}
\newcommand{\rM}{\mathcal{M}}
\newcommand{\bK}{\mathbb{K}}
\newcommand{\bQ}{\mathbb{Q}}
\newcommand{\bR}{\mathbb{R}}
\newcommand{\bZ}{\mathbb{Z}}
\newcommand{\bN}{\mathbb{N}}
\newcommand{\bC}{\mathbb{C}}
\newcommand{\bP}{\mathbb{P}}
\newcommand{\bF}{\mathbb{F}}

\ifpdf
\DeclareGraphicsExtensions{.pdf, .jpg, .tif}
\else
\DeclareGraphicsExtensions{.eps, .jpg}
\fi

%%%%%%%%%%%%%%%%%%%%%%%%%%%%%%%%%%%%%%%%%%%%%%%%%%%%%%%%
\pagestyle{empty} %Pas de numérotation

\noindent \textsc{Institution Sainte-Marie}\\
Colle de Mathématiques n°3 (XMP)\\
Lundi 28 septembre 2015

% Sources : 
% [1] Les contre-exemples en mathématiques - Ellipses
% [2] Exercices d'analyse - Armand Colin

\subsection*{\underline{Sujet 1}}

\paragraph*{Question} % [1] 10.6

Exhiber une fonction convexe qui n'est pas continue.

\paragraph*{Exercice (\emph{Inégalité de Minkowsky)}} % [2] 3.18

Soient $a_1, ..., a_n, b_1, ..., b_n \in \bR_+^*$ et $p,q>1:\frac{1}{p}+\frac{1}{q}=1$. Montrer que 
$$\left(\sum\limits_{i=1}^n(a_i+b_i)^p\right)^{1/p}\leq\left(\sum\limits_{i=1}^na_i^p\right)^{1/p}+\left(\sum\limits_{i=1}^nb_i^p\right)^{1/p}.$$
On pourra supposer l'\emph{inégalité de Hölder} (\emph{cf.} \uline{Sujet 3}) : $\sum\limits_{i=1}^na_ib_i\leq\left(\sum\limits_{i=1}^na_i^p\right)^{1/p}\left(\sum\limits_{i=1}^nb_i^q\right)^{1/q}.$

\subsection*{\underline{Sujet 2}} % [1] 10.7

Soit $f$ une fonction définie sur un intervalle réel $I$ et vérifiant 
$$\forall x,y\in I, f\left(\frac{x+y}{2}\right)\leq\frac{f(x)+f(y)}{2}.$$

On suppose qu'il existe un supplémentaire $V$ de $\bQ$ dans le $\bQ$-ev $\bR$ (conséquence de l'\emph{axiome du choix}). $f$ est-elle convexe ?

\subsection*{\underline{Sujet 3}}

\paragraph*{Question} % Programme de colle

Soit $f$ une fonction convexe sur un intervalle réel $I$, $x_1, ..., x_n$ des points de $I$, et $\lambda_1,...,\lambda_n$ des réels positifs de somme $1$. Montrer que $f(\sum\limits_{i=1}^n\lambda_ix_i)\leq\sum\limits_{i=1}^n\lambda_if(x_i))$.

\paragraph*{Exercice (\emph{Inégalité de Hölder)}} % [2] 3.18

Soient $a_1, ..., a_n, b_1, ..., b_n \in \bR_+^*$ et $p,q>1:\frac{1}{p}+\frac{1}{q}=1$. Montrer que 
$$\sum\limits_{i=1}^na_ib_i\leq\left(\sum\limits_{i=1}^na_i^p\right)^{1/p}\left(\sum\limits_{i=1}^nb_i^q\right)^{1/q}.$$
Il est possible (mais pas obligatoire) de commencer par montrer que $x,y>0\Rightarrow x^{1/p}y^{1/q}\leq\frac{x}{p}+\frac{y}{q}$.

\subsection*{\underline{Problème bonus}} % [1] 10.8

On pose successivement (avec $d$ la distance associée à la norme usuelle $|\cdot |$) :

$$\begin{array}{llll}
g : & \bR & \rightarrow & \bR\\
& x & \mapsto & d(x,\bZ),
\end{array}$$

$$\forall n \in \bN, \begin{array}{llll}
u_n : & \bR & \rightarrow & \bR\\
& x & \mapsto & \frac{g(4^nx)}{4^n},
\end{array}$$

$$\begin{array}{llll}
f : & \bR & \rightarrow & \bR\\
& x & \mapsto & \sum\limits_{n=0}^\infty u_n(x).
\end{array}$$

Montrer que $f$ n'est convexe ou concave sur aucun intervalle non-trivial de $\bR$.

(Le chapitre sur les séries de fonctions donne cependant que $f$ est continue, c'est donc un contre-exemple pour une éventuelle réciproque à $[f$ convexe sur $I$ ouvert $\Rightarrow f$ continue sur $I]$)

\vfill

\begin{center}
\url{https://github.com/m-legrand/xmp15}
\end{center}

\end{document}