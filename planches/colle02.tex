\documentclass[12pt,a4paper]{report}

\usepackage[utf8]{inputenc}
\usepackage[T1]{fontenc}
\usepackage[francais,frenchb]{babel}
\usepackage[top=1.5cm, bottom=1.5cm, left=2cm, right=2cm]{geometry}
\usepackage{lmodern}
\usepackage{amsmath}
\usepackage{amssymb}
\usepackage{graphicx}
\usepackage{mathrsfs}
\usepackage{amsthm}
\usepackage{stmaryrd}
\usepackage{verbatim}
\usepackage{moreverb}
\usepackage{textcomp}
\usepackage{hyperref} % liens avec \url
\usepackage{ulem} % pour barrer avec \sout

\DeclareMathOperator{\pgcd}{pgcd}
\DeclareMathOperator{\tr}{tr}

\begin{document}

\newcommand{\rA}{\mathcal{A}}
\newcommand{\rB}{\mathcal{B}}
\newcommand{\rC}{\mathcal{C}}
\newcommand{\rG}{\mathcal{G}}
\newcommand{\rF}{\mathcal{F}}
\newcommand{\rP}{\mathcal{P}}
\newcommand{\rH}{\mathcal{H}}
\newcommand{\rR}{\mathcal{R}}
\newcommand{\rE}{\mathcal{E}}
\newcommand{\rL}{\mathcal{L}}
\newcommand{\rM}{\mathcal{M}}
\newcommand{\bK}{\mathbb{K}}
\newcommand{\bQ}{\mathbb{Q}}
\newcommand{\bR}{\mathbb{R}}
\newcommand{\bZ}{\mathbb{Z}}
\newcommand{\bN}{\mathbb{N}}
\newcommand{\bC}{\mathbb{C}}
\newcommand{\bP}{\mathbb{P}}
\newcommand{\bF}{\mathbb{F}}

\ifpdf
\DeclareGraphicsExtensions{.pdf, .jpg, .tif}
\else
\DeclareGraphicsExtensions{.eps, .jpg}
\fi

%%%%%%%%%%%%%%%%%%%%%%%%%%%%%%%%%%%%%%%%%%%%%%%%%%%%%%%%
\pagestyle{empty} %Pas de numérotation

\noindent \textsc{Institution Sainte-Marie}\\
Colle de Mathématiques n°2 (XMP)\\
Lundi 21 septembre 2015\\

% Sources : 
% [1] Tout-en-Un MP - Dunod - Réduction des endomorphismes
% [2] Oraux X-ENS - Cassini - Algèbre 2

\subsection*{\underline{Exercice 1}} % [1] Cours (HP) + [1] 5

\paragraph{Préliminaires}
Étant donnée l'équation récurrente linaire à coefficients dans $\bC$ 
$$(E):x_{n+p}+\alpha_{p-1}x_{n+p-1}+...+\alpha_{0}x_{n}=0,$$ 
on appelle polynôme caractéristique de $(E)$ $$C(X)=X^p+\sum\limits_{k=0}^{p-1}\alpha_{k}X^k=\prod\limits_{k=1}^r(X-\lambda_k)^{p_k}.$$
En considérant $T\in\rL(\bC^{\bN}) : (u_n)_{n\in\bN} \mapsto (u_{n+1})_{n\in\bN}$ et en supposant $\alpha_0\neq 0$, montrer que toute solution de $(E)$ s'écrit $$\forall n\in \bN, x_n = \sum\limits_{k=1}^rP_k(n)\lambda_k^n$$ où les $\lambda_k$ sont les racines distinctes de $C$, et $P_k\in\bC_{p_k-1}[X]$.

\paragraph{Application}
Étant donnés $a_0, ..., a_{k-1}\in \bC$, on pose 
$\left\lbrace\begin{array}{ll}
x_0, ..., x_{p-1}&=a_0, ..., a_{p-1}\\
x_{n+p}&=\frac{1}{p}\sum\limits_{k=0}^{p-1}x_{n+k}
\end{array}\right.$.\\
$(x_n)$ admet-elle une limite ?\\
Si oui, laquelle ?

\subsection*{\underline{Exercice 2}} % [1] 13

Soit $n\in\bN^*$. Montrer que $A\text{ et }B\in\rM_n(\bC)$ ont une valeur propre commune si et seulement si $\exists U\in\rM_n^*(\bC) : UA=BU$.

\subsection*{\underline{Exercice 3}} % [1] 25 = [2] 2.33

Soit $E$ un $\bK$-ev de dimension finie $n$, avec $\bK=\bR\text{ ou }\bC$. Montrer que $u\in\rL (\bC)$ est nilpotent si et seulement si $\forall k\in \llbracket 1,n\rrbracket ,\ \tr(u^k) = 0$.

\subsection*{\underline{Problème bonus}} % [2] 2.19 + [2] 2.32

\noindent a) Soit $E$ un $\bK$-ev (avec $\bK=\bR\text{ ou }\bC$) de dimension finie $n$.
\begin{enumerate}
\item Montrer que la restriction d'un endomorphisme diagonalisable à un sous-espace stable est diagonalisable.
\item Soit $\Gamma$ un sous-ensemble de $\rL(E)$ d'endomorphismes de $E$ commutant deux à deux. Montrer que les éléments de $\Gamma$ sont simultanément diagonalisables.\\
\end{enumerate}

\noindent b) Soient $A,B\in\rM_n(\bR)$ diagonalisables telles que $e^A=e^B$. Montrer que $A=B$.

\vfill

\begin{center}
\url{https://github.com/m-legrand/xmp15}
\end{center}

\end{document}