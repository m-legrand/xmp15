\documentclass[12pt,a4paper]{report}

\usepackage[utf8]{inputenc}
\usepackage[T1]{fontenc}
\usepackage[francais,frenchb]{babel}
\usepackage[top=1.5cm, bottom=1.5cm, left=2cm, right=2cm]{geometry}
\usepackage{lmodern}
\usepackage{amsmath}
\usepackage{amssymb}
\usepackage{graphicx}
\usepackage{mathrsfs}
\usepackage{amsthm}
\usepackage{stmaryrd}
\usepackage{verbatim}
\usepackage{moreverb}
\usepackage{textcomp}
\usepackage{hyperref} % liens avec \url
\usepackage[normalem]{ulem} % pour barrer avec \sout, et autres effets intéressants

\DeclareMathOperator{\pgcd}{pgcd}
\DeclareMathOperator{\tr}{tr}

\begin{document}

\newcommand{\rA}{\mathcal{A}}
\newcommand{\rB}{\mathcal{B}}
\newcommand{\rC}{\mathcal{C}}
\newcommand{\rG}{\mathcal{G}}
\newcommand{\rF}{\mathcal{F}}
\newcommand{\rO}{\mathcal{O}}
\newcommand{\rP}{\mathcal{P}}
\newcommand{\rH}{\mathcal{H}}
\newcommand{\rR}{\mathcal{R}}
\newcommand{\rE}{\mathcal{E}}
\newcommand{\rL}{\mathcal{L}}
\newcommand{\rM}{\mathcal{M}}
\newcommand{\bK}{\mathbb{K}}
\newcommand{\bQ}{\mathbb{Q}}
\newcommand{\bR}{\mathbb{R}}
\newcommand{\bZ}{\mathbb{Z}}
\newcommand{\bN}{\mathbb{N}}
\newcommand{\bC}{\mathbb{C}}
\newcommand{\bP}{\mathbb{P}}
\newcommand{\bF}{\mathbb{F}}

\ifpdf
\DeclareGraphicsExtensions{.pdf, .jpg, .tif}
\else
\DeclareGraphicsExtensions{.eps, .jpg}
\fi

%%%%%%%%%%%%%%%%%%%%%%%%%%%%%%%%%%%%%%%%%%%%%%%%%%%%%%%%
\pagestyle{empty} %Pas de numérotation

\noindent \textsc{Institution Sainte-Marie}\\
Colle de Mathématiques n°6 (XMP)\\
Lundi 2 novembre 2015

% Sources : 
% Oraux X-ENS - Analyse 1 - Chapitre 3

\subsection*{\underline{Sujet 1}}
Soit $(a_n)_{n\in\bN^*}$ une suite de réel positifs non nuls telle que la série $\sum a_n$ converge.
\begin{enumerate}
\item Montrer que si $\alpha>\frac{1}{2}$, la série $\sum\frac{\sqrt{a_n}}{n^\alpha}$ converge.
\item Que dire du cas $\alpha=\frac{1}{2}$ ?
\end{enumerate}

\subsection*{\underline{Sujet 2}}
Soit $(u_n)_{n\in\bN^*}$ une suite positive décroissante convergeant vers $0$. Montrer que les séries $\sum u_n$ et $\sum n(u_n-u_{n+1})$ sont de même nature et ont, en cas de convergence, la même valeur.

\subsection*{\underline{Sujet 3}}
Déterminer la nature de la série $\sum \ln\left(\frac{(\ln(n+1))^2}{\ln(n)\ln(n+2)}\right)$.

\subsection*{\underline{Sujet bonus}}
Soit $(a_n)_{n\in\bN^*}$ une suite strictement positive telle que $\sum a_n$ converge, et $(b_n)_{n\in\bN^*}$ une suite réelle vérifiant $b_n = 1+O\left(\frac{1}{\ln(n)}\right)$. Étudier la nature de la suite $\sum a_n^{b_n}$.

\vfill

\begin{center}
\url{https://github.com/m-legrand/xmp15}
\end{center}

\end{document}